\documentclass[sthlmFooter, 10pt]{beamer}
\usetheme{sthlm}
\usepackage[fnsymbol, geometry]{jipkg}

\title{The Relationship Equation}
\subtitle{Subject inspired by Hannah Fry}
\author{Cléo \textsc{Daguin} \& Quentin \textsc{Ribac}}
\date{\today}
\institute{IUT of Lannion, English class}

\begin{document}
\selectlanguage{english}
\maketitle

\begin{frame}{Introduction}
\alert{How can math quantify and model emotions in a relationship?}
\begin{block}{The Relationship Equation}
$
W_{t+1} = w + r_w W_t + I_{HW}(H_t)
$
\end{block}
\end{frame}

\begin{frame}{Tools}
\begin{center}
% HTML, CSS, bootstrap (Cléo)
% GitHub, JS, LaTex (Quentin)
\includegraphics[width=0.3\textwidth]{quad_logos.png}
\includegraphics[width=0.3\textwidth]{github.png}

And \alert{\LaTeX} for this \alert{presentation} and \alert{math} on the website.
\end{center}
\end{frame}

\begin{frame}{Content tree}
\begin{itemize}
\item Source and idea (Hannah Fry)
\item The Equation explained
\item The \textsc{Spaff} system
\item Shapes of graphs
\item About
\end{itemize}
\end{frame}

% source and idea (Hannah Fry -> TED, book, numberphile) %C
% The equation --> explanation of each term %Q
% Spaff (emotion rating) %C
% graphs of different shapes (t->W \& t->H, H->W) %Q
% about %C

\section{Questions?}

\end{document}

